\documentclass[a4paper,12pt]{article}
\synctex=1
\pdfoutput=1

\usepackage[utf8]{inputenc}
\usepackage[T1]{fontenc}
\usepackage{lmodern}
\usepackage{microtype}
\usepackage{minted}
\usepackage{parskip}

\begin{document}

\section*{Sliding window for minimum}

Computing the minimum values of a sliding window is not as easy as computing,
e.g., the sum.  When computing the sum, we simply need a single accumulated
variable \mintinline{python}{acc = acc - data[win] + data[win + k]}.

However, when we want to compute the minimum value, we need to know which next
minimum value to keep when a small value disappears out of the window.
%
Recomputing this value every iteration is more expensive than necessary; It
would take $O(n \cdot k)$, where $n$ is the number of elements and $k$ the size
of the window.  There is, however, a way of computing all min-values in $O(n)$
time.

The trick is to use a \emph{double-ended queue}.  In this queue, we keep a
(necessarily increasing) list of potential minimum values.  To understand the
algorithm, consider the following list: \mintinline{python}{data = [10, 5, 3, 5, 7, 4, 2]}, with $k = 3$.

Let \mintinline{python}{Q = []} be an empty double-ended queue%
\footnote{A double-ended queue is one we can read and write, push and pop, from both left and right side in $O(1)$ time.}%
, and let us start reading values from \mintinline{python}{data}.
%
First we read 10, which we will put into the queue.  Second, we read 5.  Notice
that 10 is now not relevant anymore, since 10 will never be the minimum value.
We will pop \emph{from the right} until the queue is empty, or until the
right-most value is higher than what we are currently looking at.  The queue,
after having read $10, 5$ is \mintinline{python}{Q = [5]}.

We still haven't read $k=3$ values yet, so we read another one, and see 3.
Again, we have that $5$ is not relevant, so we remove it from the queue.  After
having read $10, 5, 3$, we have \mintinline{python}{Q = [3]}.

Since we have seen $k=3$ values, we can output $3$.  The window is now
at $$[\langle 10, 5, 3 \rangle 5, 7, 4, 2].$$ At this point, we are ready to
move the window one step to the right, i.e., we want to discard 10 and
introduce 5.  Since the left-most value in \mintinline{python}{data} is 10, and
10 is not in the queue, we do not pop anything from the queue.  However, the
value we include in the window might at some point in the future become the
minimum value (after 3 has been discarded some time in the future), so we
include 5 in the queue.

At $[10, \langle 5, 3, 5 \rangle 7, 4, 2]$ with
%
\mintinline{python}{Q = [3, 5]},
%
we again output the minimum value; it is the first value in the queue.
%
Moving the window one step to the right means we discard a value 5.  However,
the leftmost value in the queue is 3, which means that the value we discard is
irrelevant.  On the right hand side, we include 7.  This might in the future
become the minimum value, so we need to include it in the queue.

The state is now $[10, 5 \langle 3, 5, 7 \rangle 4, 2]$ with
%
\mintinline{python}{Q = [3, 5, 7]}, and we again output a value: the leftmost
value in the queue, 3.  Thus far, our output has been
%
\mintinline{python}{output = [3, 3, 3]}.

Finally, something interesting happens, we are ready to discard the value 3.
Moving the window one step to the right means we discard a value that is the
leftmost value in the queue, which means we need to pop it off.  The queue is
then
%
\mintinline{python}{Q = [5, 7]},
%
but we also need to include the new value, which happens to be 4.

When the queue is \mintinline{python}{Q = [5, 7]}, and the new value is 4, we
can notice that neither 5 nor 7 will ever be output, because (a) 4 is lower
than them both and (b) 4 will outlive them.  The procedure is to look at the
rightmost value, and since $7 > 4$, we pop off 7, and repeat.  The rightmost
value is $5 > 4$ and we will pop off 5, and we are left with an empty queue,
and therefore the state is:

$$[10, 5 ,3 \langle 5, 7, 4 \rangle  2],$$
with
%
\mintinline{python}{Q = [4]}
%
and we output the leftmost value in the queue: 4.

The final state is $[10, 5 ,3 ,5 \langle 7, 4 , 2 \rangle]$, with
%
\mintinline{python}{Q = [2]}
%
at which point we output 2.  The total output is
%
\mintinline{python}{output = [3, 3, 3, 4, 2]}.
%

\appendix
\section*{Code}

\begin{minted}{python}
def slidemin(data, k):
    """Linear time algorithm for computing sliding
       window minimum values."""
    minq = deque()
    for i in range(len(data)):
        if minq and minq[0] == i - k:
            minq.popleft()
        while minq and data[minq[-1]] >= data[i]:
            minq.pop()
        minq.append(i)
        if i >= k - 1:
            yield data[minq[0]]
\end{minted}
\end{document}

% Local Variables:
% TeX-command-extra-options: "-shell-escape"
% End:
